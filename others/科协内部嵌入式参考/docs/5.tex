
\chapter{基础篇最终话: 随心操控的小灯}

在这个实验中, 我们将利用前面讲到的UART串口通讯, 让小灯“按着我们的意思”来闪烁. 请观看视频
\href{https://www.bilibili.com/video/BV1bc411J7Tv/?p=10&share_source=copy_web&vd_source=00b9d329964a93c9843f9c524074f948}{串口中断}, 尝试利用串口通讯实现
通过在键盘上输入不同数字, 实现改变灯的亮灭.

\section{缓冲区与消息队列}
在中断机制中, 缓冲区和消息队列是两种常用的数据结构, 它们用于管理和存储中断相关的数据.
\subsection{缓冲区}
缓冲区是内存中的一块区域, 用于临时存储数据. 在中断上下文中, 缓冲区通常用于存储从硬件设备(如储存介质, 键盘等)传入的数据, 或者存储准备发送到硬件设备的数据. 当硬件设备准备好接收或发送数据时, 它会发出中断信号, 操作系统将数据从设备的缓冲区复制到内核的缓冲区, 或者从内核的缓冲区复制到设备的缓冲区.

一种特殊的缓冲区叫环形缓冲区, 它在数据结构上首尾相连, 形成一个环状, 因此得名(实际储存在内存中当然还是线性的). 环形缓冲区环形缓冲区在创建时被分配为一个固定大小的数组, 即它有一定的容量限制. 它用两个指针来跟踪数据的写入和读取位置. 写指针指示下一个数据应该被写入的位置, 而读指针指示下一个数据应该被读取的位置.
如果写指针追上读指针, 新的数据将覆盖旧的数据, 或者也可以定义一些其他的处理策略. 但普通的线性缓冲区写满后就无法再写入了.
相比普通线性缓冲区, 环形缓冲区一个很大的优势就是数据被写入和读取时, 不需要移动其他数据, 大大减少了内存操作的次数.
\subsection{消息队列}
消息队列是一种数据结构, 它允许消息按顺序存储和检索. 在中断机制中, 消息队列用于管理中断请求和中断处理程序的通信. 当一个中断发生时, 它可以向消息队列中添加一个消息, 该消息包含了中断的详细信息. 操作系统或中断服务例程可以按顺序处理消息队列中的消息, 确保每个中断都得到适当的响应.

消息队列的好处是它提供了一种有序的方式来处理多个中断, 可以确保高优先级的中断得到优先处理, 并且不会因为低优先级的中断而延迟.

\section{指令协议}

在更复杂的应用场合, 我们会接触到不同的指令协议. 它们规定了传输数据的格式和规范. 例如一个指令的起始和结束符, 以及各个指令中, 哪几个byte代表什么样的数据. 在这里我们就为板子上的LED规定一个属于我们自己的指令协议.

为了方便解析指令, 我们使用反斜杠和分号作为指令的起始和结束符. 指令的内容分为两部分: \verb|identifier| 和 \verb|dest| , 分别代表操作的类型和操作的对象. 目前我们定义两个 \verb|identifier|, \verb|LTON| 开灯, \verb|LTOFF| 关灯.

\begin{lstlisting}
	\LTON 1;
	\LTOFF 3;
\end{lstlisting}

在用串口发出上述指令之后, 分别会见到LED 1亮和LED 3灭的现象. 如果遇到了未知指令, 则STM32使用UART回复: \verb|Unknown identifier.|

\section{创建工程和获取框架代码}

请按照之前的方法创建工程, 并且使用CubeMX初始化好UART, GPIO等等外设和相应的中断设置.

工程创建完毕后, 在 \verb|src| 文件夹中下载需要的框架代码. 你可以使用以下任何一种方法:

\begin{itemize}
	\item 使用git(推荐): \begin{lstlisting}
		git clone https://github.com/YeonGu/ex1.git
	\end{lstlisting}
	\item 直接下载: 访问 \href{https://github.com/YeonGu/ex1}{https://github.com/YeonGu/ex1}, 通过Code -> Download Zip的方式下载代码, 然后解压到src/文件夹中.
	\item 任何能够把GitHub上的代码偷下来的方法.
\end{itemize}


\section{修改历史}

2024/3 完成框架代码. (顾雨杭)

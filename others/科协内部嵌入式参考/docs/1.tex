\chapter{开始之前: 准备工作与相关注意事项}

在本章中, 你将会完成嵌入式开发相关的准备工作 --- 包括相关软件和环境的安装.

% \section{准备工作}
\section{开发环境配置}

\begin{theorem}{中文路径与用户名}
	在嵌入式开发的过程中, 一定要避免中文路径和Windows用户名 (C://Users  下的文件夹) 的使用!

	否则你在本节的安装过程中就会遇到令人费解的问题.

	如果你的Windows的用户名已经设置成了中文, 请参考网上的资料对C://Users下的文件夹名称进行更改.
\end{theorem}

\subsection{什么是开发环境?}

所谓开发环境就是用于开发的一系列工具 --- 代码编辑器, 代码分析器, 编译器 (当然, 还有嵌入式开发的下载器).

\begin{definition}{Visual Studio 的本质}
	在大一的课程中, 只需要安装好VS, 点两下鼠标, 就可以创建工程, 编写代码, 构建项目, 运行程序. 我们把Visual Studio称为"IDE", 就是因为它的本质是一套集成开发环境.

	现在让我们对Visual Studio怯魅 --- 仅仅使用文本编辑器和系统命令, 应该怎样编写, 编译和运行一个C/C++程序?

	你可以去之前提到的笨方法学C中寻找答案.
\end{definition}

\subsection{配置CubeIDE作为开发环境}

如果你习惯于使用Keil, 那你可以忽略本节. 尽管Keil作为嵌入式开发IDE方面的权威备受工业界推崇, 但它无论是从界面还是代码编写体验上讲都实在不是一个合格的现代IDE.

这里我们推荐使用的是ST公司专为stm32开发定制的IDE, \textbf{STM32CubeIDE}.

请按照\href{https://b23.tv/ieviM3F}{安装开发环境 STM32CubeIDE | keysking的stm32教程-哔哩哔哩}视频中的指引安装好CubeIDE.

\subsection{*配置CLion作为开发环境}

CLion是Jetbrains推出的C/C++ IDE, 以其友好的界面, 强大的代码提示, 高效的代码重构工具而闻名. 如果你愿意为更好的开发体验而折腾, 不妨尝试为CLion添加STM32开发支持. CLion在2019年起就开放了对STM32开发的官方支持.

配置的方法可以寻找稚晖君的知乎文章, 以及一些其他的博客文章.

\begin{theorem}
	STM32CubeMX 在6.5版本之后无法生成CLion所需要的SW4STM32类型项目, 在安装时可以到官网选择旧版本, 或是参考\href{https://blog.csdn.net/m0_54490453/article/details/128921674}{本文章}.
\end{theorem}


\section{熟悉全新的世界}

在配置完开发环境后, 不要呆在那里, 先试着熟悉一下环境吧.

\begin{itemize}
	\item 生成一个新项目(选择什么型号/开发板?)
	\item 生成的项目十分庞大, 它的文件组织结构大致是什么样的?
	\item 之后需要在这里编写代码, 那么代码应该写在哪里? (程序设计课的时候代码是写在哪里的?)
	\item 激动人心的时刻: 点击构建(Build), 编译项目试试看吧!
\end{itemize}

\section{开发板型号以及相关参考}

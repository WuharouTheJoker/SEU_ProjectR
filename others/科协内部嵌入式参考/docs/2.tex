\chapter{点灯工程师}

点灯之于嵌入式开发的学习就如同Hello World之于编程的学习, 几乎是每位学习者迈出的第一步. 在这个过程中, 你将会掌握对MCU上最基本的外设 --- GPIO的相关操作, 延时函数的使用, 以及整个C代码工程的整体情况.

\section{点亮你的LED}

\subsection{创建工程}

% Board selector (参考实验指导书)
我们所使用的Nucleo-144开发板是ST官方推出的, 基于STM32F413ZH的开发板. 因此STM32CubeMX软件中可以直接选用这个板子, 数据库会帮我们初始化一些引脚信息而不用自己手动配置.

关于创建工程, 初始化板子型号的步骤可以参见\textbf{STM32实验指导书}中\textbf{第四章 跑马灯实验}的内容.

\subsection{GPIO}

\begin{definition}
	A general-purpose input/output (GPIO) is an uncommitted digital signal pin on an integrated circuit or electronic circuit (e.g. MCUs/MPUs) board which may be used as an input or output, or both, and is controllable by software.

	--- Wikipedia
\end{definition}

\subsection{编写代码}

% 了解设置和定义宏的位置
% board 和 mcu 两种配置

请先观看视频: \href{https://www.bilibili.com/video/BV1s84y1h77Q/?share_source=copy_web&vd_source=b25682b699c98b1c6e3947c9b7b66a3a}{点亮第一颗小灯}和\href{https://www.bilibili.com/video/BV1Cd4y1L7kH/?spm_id_from=333.788&vd_source=f1ae41cbb2101ee1a2a77931d719f1be}{闪烁的小灯}. 完成视频里所演示的代码的内容. 注意: 由于型号和开发板不同, 创建工程的过程请先参考实验指导书第四章.

\begin{definition}
	视频中展示的点灯所用到的GPIO引脚和我们使用的开发板并不一致. 在使用相应API的时候示例如下:

	\lstset{language=C}
	\begin{lstlisting}
		HAL_GPIO_WritePin ( LD1_GPIO_Port, LD1_Pin, GPIO_PIN_SET ); // 将LD1对应的GPIO设为高电平
		HAL_GPIO_WritePin ( LD2_GPIO_Port, LD2_Pin, GPIO_PIN_SET ); // 将LD2对应的GPIO设为高电平
	\end{lstlisting}

\end{definition}

% TODO: 注释的内容

\section{RTFSC (1)}

\subsection{引脚和宏的使用}
% 注释
让我们把目光首先放在LD1\_Pin这几个类似的定义上. 很明显, LD1\_GPIO\_Port, LD1\_Pin, GPIO\_PIN\_SET都不是本来就属于C的东西, 那么它们是如何出现的呢?

使用IDE的代码跳转功能可以轻松地帮助我们解答这个疑问. 哦! 原来它们的奥妙在CubeMX为我们生成的源码里:
\begin{lstlisting}{main.h}
	// in main.h...
	/* Private defines -----------------------------------------------------------*/
	#define USER_Btn_Pin GPIO_PIN_13
	#define USER_Btn_GPIO_Port GPIOC
	#define LD1_Pin GPIO_PIN_0
	#define LD1_GPIO_Port GPIOB
	// ......
\end{lstlisting}

接着, main.c引用了main.h, 于是我们可以直接在代码里写LD1\_Pin; 在编译期间, 这些定义的宏会被自动展开成本应该有的样子.

\begin{problem}
如果你对\textbf{预编译指令, \#define, \#include} 这些概念仍不熟悉的话, 是时候STFW了解一下了.

以及: \#include 的本质其实是"复制粘贴". 怎么理解这个概念?

它将有助于你解释你可能经常遇到的undeclared identifier, multiple declaration 之类的问题, 以及为什么每一个头文件都需要有一个令人费解的\#ifndef...\#define...\#endif.
\end{problem}
\vspace{15pt}
\hrule
\vspace{10pt}
通过C中的宏, 我们实现了一种软件的封装. 我们当然可以把之前的代码写成:
\begin{lstlisting}
	HAL_GPIO_WritePin ( GPIOB, GPIO_PIN_0, 1 ); // 将LD1对应的GPIO设为高电平
	HAL_GPIO_WritePin ( GPIOB, GPIO_PIN_2, 1 ); // 将LD2对应的GPIO设为高电平
\end{lstlisting}

但这种写法很大程度上降低了代码可读性, 更重要的是破坏了\textbf{可维护性}: 如果在未来这个引脚被改变了, 这是否代表着我们未来需要一个一个改引脚?

而我们的写法则没有这种顾虑: 只需要更改 \#define 的内容, 剩下的任务编译器会在启动编译时的预编译期间解决一切问题.

\subsection{引脚的配置}

接下来我们会在硬件, CubeMX, 代码三个层面理解引脚和配置. 阅读开发板原理图, 可以发现LD1被接到了PB0这个GPIO引脚.

那么代码中呢? 在上一节, 我们已经看到了:
\begin{lstlisting}
	#define LD1_Pin GPIO_PIN_0
	#define LD1_GPIO_Port GPIOB
\end{lstlisting}

我们定义的LD1对应的GPIO刚好是GPIOB端口的PB0! 同样地, 请你寻找开发板原理图上的 LD2, 3, 寻找它们对应的引脚和宏声明.

那么CubeMX是如何知道这一点的呢? 别忘了我们生成项目的时候勾选了 "default initialize peripheral", 由于官方的开发板在数据库中已经有了LED端口的信息, 它们在一开始就已经被配置好了. 你可以打开CubeMX, 在Pinout\&configuration一栏中寻找到这几个GPIO, 看看它们是如何被定义好的.

\begin{theorem}
	了解了引脚的配置, 接下来请你\textbf{使用MCU selector} 而不是 Board selector 重新生成一个工程, 自己为LD0, 1, 2进行引脚的定义, 并点亮这三颗LED. 指定引脚的过程可以看 2.1.3 部分中的视频.
\end{theorem}

\section{流水灯}

到目前为止, 你已经学会了基础的点灯, 以及结合视频完成了动态变换的LED效果. 接下来请你自己完成一个流水灯: 使LD1, LD2, LD3按顺序交替点亮.

\subsection{copy-paste}

到这一步, 估计你的代码已经变成了这样:
\begin{lstlisting}
	HAL_GPIO_WritePin ( LD1_GPIO_Port, LD1_Pin, GPIO_PIN_SET ); // 将LD1对应的GPIO设为高电平
	HAL_GPIO_WritePin ( LD2_GPIO_Port, LD2_Pin, GPIO_PIN_RESET ); // 将LD2对应的GPIO设为低电平
	HAL_GPIO_WritePin ( LD3_GPIO_Port, LD3_Pin, GPIO_PIN_RESET ); // 将LD3对应的GPIO设为低电平
	HAL_Delay(500); // 延时, 点亮LD2
	HAL_GPIO_WritePin ( LD1_GPIO_Port, LD1_Pin, GPIO_PIN_RESET ); // 将LD1对应的GPIO设为低电平
	HAL_GPIO_WritePin ( LD2_GPIO_Port, LD2_Pin, GPIO_PIN_SET ); // 将LD2对应的GPIO设为高电平
	HAL_GPIO_WritePin ( LD3_GPIO_Port, LD3_Pin, GPIO_PIN_RESET ); // 将LD3对应的GPIO设为低电平
	// ......
\end{lstlisting}

--- 丑到家了, 对不对? 那有没有什么办法让我们要写的东西少一点, 代码好看一点呢?

当然是有的 --- 那就是我们之前提到的\textbf{宏}.

再加上一点点小技巧 --- (\textbf{宏拼接和宏参数}) --- 我们就可以把"点灯"的操作\textbf{封装}起来.
\begin{lstlisting}
	#define LTON(n) HAL_GPIO_WritePin ( LD##n##_GPIO_Port, LD##n##_Pin, GPIO_PIN_SET ) // 点亮LDn
	#define LTOFF(n) HAL_GPIO_WritePin ( LD##n##_GPIO_Port, LD##n##_Pin, GPIO_PIN_RESET ) // 熄灭LDn
\end{lstlisting}

特别注意其中\#\#的\textbf{拼接宏}用法(如果难以理解, 可以STFW). 于是原本的代码就可以写成:
\begin{lstlisting}
	LDON(1);
	LDOFF(2);
	LDOFF(3);
	HAL_Delay(500); // 延时, 点亮LD2
	LDOFF(1);
	LDON(2);
	LDOFF(3);
\end{lstlisting}

好看很多了, 对吧? 不过每次点亮一个灯就要写三行疑似还是有点太臭了, 让我们在此基础上再套一层宏(使用\textbf{反斜杠可以定义多行的宏}):
\begin{lstlisting}
	#ONELT(n)     \      // 点亮LDn, 熄灭其他LED
		LDOFF(1); \
		LDOFF(2); \
		LDOFF(3); \
		LDON(n);
\end{lstlisting}

于是:
\begin{lstlisting}
	ONELT(1);
	HAL_Delay(500);
	ONELT(2);
\end{lstlisting}

\begin{theorem}
	请用宏定义等方法重构自己的代码, 创造一个属于你的炫酷点灯程序吧!
\end{theorem}

\begin{problem}
为什么要费尽心思设计这些宏? 请看本节的小标题: Copy-paste --- 指的就是本节一开始, 我们写出来的代码的样子.

经过这一番代码层面的优化, 相信聪明的你已经get到了为什么Copy-paste行为是写代码的大忌, 以及我们有哪些规避写出Copy-paste代码的手段.

当然, 解决copy-paste问题的方案不仅仅有宏, 更复杂的任务也可以交给函数去完成 --- 不过由于宏仅仅是文本展开, 而不涉及代码跳转, 毫无疑问它是不会造成多余的计算资源消耗的. (不过宏会导致最终编译的代码所需的空间更大; 但是相比起嵌入式开发中金贵的计算资源而言, 这点空间的使用大部分时候算不上什么)
\end{problem}

\begin{definition}
	还有一种有趣的宏用法: 可变参数宏. 比如:
	\begin{lstlisting}
		#define CASE(n, ...) \
		    case n:          \
		        __VA_ARGS__; \
		        break;
	\end{lstlisting}
	于是, 我们可以把switch写成这样:
	\begin{lstlisting}
        switch (state) {
            CASE(0, ONELT(0));
            CASE(3, ONELT(1));
            CASE(4, ONELT(2));
            default:
                break;
        }
    \end{lstlisting}
	你可以STFW, 来弄明白发生了什么; 或者还有一种好方法 --- 为什么不问问神奇的ChatGPT呢?
\end{definition}
\begin{definition}
	复杂的宏也许会让人一头雾水. CLion之类的现代IDE可以给出宏展开的结果; 但是一旦宏复杂起来, 就连强大的IDE也无法告诉你它本来是什么样的, 还会影响它正常的代码提示和跳转. 在这种情况下应当如何理解复杂的宏?

	想想看宏是如何被编译器所理解的 --- 它会在编译的第一步, 预处理阶段, 被编译器展开.

	那么如果我们可以让编译器(如gcc)输出源代码预处理的结果, 不就可以弄明白发生了什么了吗?
\end{definition}

\section{RTFM}
刚才我们已经熟悉了如何使用代码操作GPIO. GPIO是MCU上最基本和重要的外设, 操作输出的电平, 从而控制设备状态, 传输数字信息.

因此, 接下来的内容会继续深入GPIO的内容, 不仅仅是为了熟悉这个外设的使用, 也是希望能给大家讲解一种MCU的外设工作的范式, 以及最重要的 --- 手册的阅读和用正确的方法解决问题.


\subsection{GPIO: 外设, 时钟和基本情况}

\subsubsection{GPIO}
了解单片机上外设的基本情况最好的资料就是官方的\textbf{参考手册 (Reference manual)}. 对于一个MCU, 一般会有\textbf{数据手册 Datasheet}和\textbf{参考手册 Reference manual}两种较为重要的官方参考. 前者一般说明了MCU的电气特性, 封装规格等等物理信息; 后者则更像是使用人员的参考手册, 详细说明了MCU各个外设的使用情况和控制方式等, 用于编程参考.

接下来就让我们尝试阅读\textbf{参考手册}, 了解单片机上GPIO的基本情况吧.

\begin{theorem}{RTFM}
	打开stm32f413的参考手册(官网下载或是用群里发的), 尝试:
	\begin{enumerate}
		\item 找到和GPIO有关的内容.
		\item 控制GPIO的寄存器都有哪些? 它们都有什么样的用途? (关于"寄存器"和"控制", 之后的内容会进一步阐释)
		\item GPIO都有一些什么特性? 它们大致都是什么意思?
		\item GPIO都有什么样的功能?
		\item 目前我们所用到的GPIO输出模式仅有"推挽"输出(Push-pull), 然而手册中还提到了许多其他的模式, 尝试搜索一下, 理解它们是什么意思. (也可以参考视频: \href{https://www.bilibili.com/video/BV1D84y1c7GV/?share_source=copy_web&vd_source=b25682b699c98b1c6e3947c9b7b66a3a}{【推挽, 开漏, 高阻, 这都是谁想出来的词??\ 】 }) (你同样能在手册中看到描述基本结构的电路图, 大二的同学应当能理解大部分了. 你可以注意到其中推挽的部分 --- 这不就是我们数电课上学过的CMOS结构嘛!)
	\end{enumerate}
\end{theorem}

\subsubsection{外设}
让我们介绍一下外设的概念.
\begin{definition}
	A peripheral device, or simply peripheral, is an auxiliary hardware device used to transfer information into and out of a computer.

	The term peripheral device refers to all hardware components that are attached to a computer and are controlled by the computer system, but they are not the core components of the computer.

	--- Wikipedia
\end{definition}

也就是说, 外设是这样一种计算机系统的组成部分:
\begin{itemize}
	\item 地位: 依附于计算机, 但不是core(处理器核心)的一部分.
	\item 控制手段: 它可以被计算机系统所控制.
	\item 用途: 输入/输出信息.
	\item 物理特性: 挂载于计算机总线(这个之后会进行解释).
\end{itemize}

因此, GPIO就是一种外设: 它被计算机所控制, 用来输出/输入电平信息.

那么, MCU是如何操控GPIO外设的呢?



\subsection{GPIO: 寄存器控制 --- 外设即访存, 与计算机系统的抽象}

请先阅读本文章的第一节"输入输出"中的内容: \href{https://nju-projectn.github.io/ics-pa-gitbook/ics2023/2.5.html}{设备与输入输出}, 理解计算机系统中"控制设备"意味着什么.

于是你可以理解了: 对于计算机系统而言, \textbf{外设的控制即访存}. 详细一点地说, 控制一个外设的过程是这样的:
\begin{itemize}
	\item \textbf{CPU核心(C代码):} 向外设对应的地址读/写数据, 和普通的变量和指针的操作没有区别.
	\item \textbf{总线}: CPU发出的向内存读/写数据的请求会经过\textbf{总线}处理, 发送到它实际映射到的位置(RAM运行内存, 程序存储器, 或是\textbf{外设}等). 对MCU上的外设 (如GPIO) 而言, 这些数据就会去到地址对应的\textbf{寄存器}, 并且读取/更改寄存器中的数据.
	\item \textbf{外设模块}: 这些寄存器中的数据便是控制外设的根据. 寄存器中的每一个bit都被赋予了特定的意义, 外设即会根据这些控制字进行工作.
\end{itemize}

\subsection{属于你的WritePin()}

说到这里, 让我们做个有趣的任务: 实现属于自己的WritePin()吧. 请按照之前的开发板配置方法建立新工程, 此时三个GPIO灯的模式已经被配好(如果你对配置过程感兴趣, 可以RTFM看看GPIO端口配置寄存器相关的内容). 在这里我们只需要通过手操寄存器来实现单纯的点灯操作就可以了.

\begin{definition}{寄存器在哪里?}
	很快你便碰到了第一个问题. 之前我们说过, 外设即访存, 那直接用指针写一堆数据不就行了嘛 --- 不过我们连相关寄存器的地址和用处都不知道, 那怎么办呢?

	别忘了, 我们之前已经阅读过Reference Manual了, 这个问题的答案也同样可以在手册中找到.


\end{definition}

\begin{theorem}{寄存器的定义}
	GPIO相关寄存器种类繁多. 不过Cubemx已经帮我们做好了初始化的工作, 我们只需要关注和输出电平控制有关的寄存器定义就可以了.

\end{theorem}
因此你阅读了手册目录, 发现目前需要关注的只有\textbf{GPIO 端口输出数据寄存器 (GPIOx\_ODR) (x = A...H)}和\textbf{GPIO 端口置1/复位寄存器 (GPIOx\_BSRR) (x = A...H)}两个寄存器 (其他寄存器都是做什么用的呢?).

其中, ODR寄存器可以进行读写, 直接设置GPIOx的每一位引脚的输出电平(ODR中只有低16位有作用, 对应着一个GPIO端口上的16个引脚); BSRR寄存器则为只写, 在写BSRR时, 相应位对应的GPIO引脚输出直接进行置位/复位.

\begin{theorem}
	阅读手册中ODR和BSRR寄存器的描述, 理解这两个寄存器中每一位的功能定义.
\end{theorem}
\begin{definition}
	BSRR的只写功能与我们惯常情况下所理解的"访存(读写内存)"有所不同(如果读BSRR, 只会获得0x0的值). 可以这么理解, \underline{访存是一种对外设的抽象}, 在读写外设的场景下, 它不仅是单纯的"读写内存", 而更像是对外设功能的一种抽象描述.
\end{definition}
\begin{definition}{寄存器地址在哪里?}
	开发板上的LED都连接在GPIOB端口. 所以我们只需要找GPIOB\_ODR和GPIOB\_BSRR的地址就可以了.

	首先我们需要找到GPIOB的\textbf{基地址}: 所有GPIOB的相关寄存器地址都会在这个基础上加上一个偏移量得到. 偏移量在寄存器的功能说明中已经给出.
\end{definition}

在手册的第二章2.2存储器组织架构一节, 你可以找到GPIOB的地址范围; 同时, 你也可以看到各种其他内存段的定义和声明, 有兴趣的话可以翻一翻里面有什么. 这里也有一些令人费解的新名词: AHB, APB等等, 它们都是总线的类型, 区别在于性能和功耗.

相信到这里, 你已经差不多理解了"外设即访存"究竟是什么意思.

\subsubsection{实现}
接下来就来实现手操寄存器点灯的功能吧.

\begin{lstlisting}
#define GPIOB_BASE 0x ? ? ? ? ? ? ? ?         // To be implemented...
#define GPIOx_ODR_OFFSET 0x ? ? ? ? ? ? ? ? ? // To be implemented...
#define GET_ADDR(base, offset) (base + offset)
#define GPIOB_REG(reg) volatile (uint32_t *)(GET_ADDR(GPIOB_BASE, GPIOx_##reg##_OFFSET))

/**
 * @brief GPIOB的PBn引脚置位
 */
void gpiob_set(uint8_t n) {
  // To be implemented...
}
/**
 * @brief GPIOB的PBn引脚复位
 */
void gpiob_reset(uint8_t n) {
  // To be implemented...
}
/**
 * @brief GPIOB的PBn引脚写电平, 如果val==0则复位, 否则置位
 */
void gpiob_write(uint8_t n, uint8_t val) {
  // To be implemented...
}
\end{lstlisting}

这里提一下位操作的方法: 如果想把某个数据的某一个bit置为0, 可以像这样做:
\begin{lstlisting}
	uint8_t target;
	// 将target的第n位设置为0
	target = (~(1 << n)) & target;
\end{lstlisting}

也就是说, 生成一个除了第n个bit以外都是1的数, 然后将它和target进行AND操作, 那么仅有的那个0就会将target的对应位设置为0. (0和任何数相与结果都为0)

这样的一个"除了第n个bit以外都是1的数"叫做Mask(遮罩), 它说明了数据中的哪些位是为我们所需要的.

同样地, 将某一位设置成1也可以用类似的手法, 通过和"1"进行的\textbf{按位或}操作完成.


\section{RTFSC (2)}
读一读HAL库的WritePin()和相关源码, 试着理解一下它是怎么实现的吧.
% GPIO相关与外设操作
